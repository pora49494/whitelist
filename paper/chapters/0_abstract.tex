It is of premise that IoT devices will become a part of human society. However, since the quality and capacity of cyber security of IoT system is considerably weak and the risk of problem like privacy intruding, spying, data stealing, occurring is considerably high. In order to alleviate the IoT system’s cyber security problem, we proposed the security system using the whitelist method with switch-level security function. The system has three operation stage: Preparation, Analyzation and Operation stage. First, system traffic data is captured during the Preparation stage. In the Analyzation stage, the system discovers all IoT devices in the system, and examine their communication pattern to extract the communication nodes listed up in the whitelist. During the Operation stage, only the packets from hosts in the whitelist are accepted. We conducted experiments to find a way to achieve and implement the functions described above. We found that the best way to discover devices in the LAN is to analyze ARP request packets observed in the LAN segment. The secure (un-malicious) hosts of each device are the hosts which our devices initiate the connection to. We have implemented filtering function using “iptables" command. The result showed that this approach worked well with the experimented IoT system, discovering all devices and extracting most of devices’ pairing secure hosts. 